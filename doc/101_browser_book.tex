\documentclass[10pt]{scrbook}
\usepackage[utf8]{inputenc}
\usepackage{ngerman}
\usepackage{amssymb}
\author{Wolfgang Keller}
\title{Documentation of 101\_browser}
\begin{document}
\maketitle
\tableofcontents

\part{Introductory material}

\chapter{FAQ}

\paragraph{What will make 101\_browser special?}

There are lots of existing browsers around. Why would you need another one?
\begin{itemize}
\item Current web browsers tend to get optimized for consuming and not for adding abilities to hack or reverse-engineer websites. Especially in areas of analyzing downloaded binary files (most browsers rely on external libraries there). So there is a gap to fill.
\item Wolfgang Keller does not believe that the browser vendors put enough emphasis on security and privacy. It is planned (if time allows) to do correctness proofs of critical parts. Additionally sometimes it is approved to risk security holes for compliance with outdated websites (for example: content negotiation or cross-side Javascript). 101\_browser will break standards in these cases (but in a very well-documented way, so that web developers won't have to fear of "`another browser hell"')
\item Many different standards (HTML+JS, PDF, SWF) tend to converge in delivered features. Why not write the browser as a unified runtime engine for all of them?
\end{itemize}

\paragraph{Where does the name "`101\_browser"' come from?}

Originally Wolfgang Keller called the browser project "`x\^{}2+1"' (he like naming his project after mathematical objects). Where does this strange name come from?

Consider the field of real numbers $\mathbb{R}$. As you learn in any introductory course about abstract algebra for this field there exists (up to isomorphism) only one algebraic field extension, which is isomorphic to $^{\mathbb{R}[x]}/_{\left\langle x^2+1\right\rangle}$, where $\left\langle x^2+1\right\rangle$ -- of course -- denotes the ideal generated by $x^2+1$.

This field extension is -- of course -- isomorphic to the field of complex numbers.

If you look back into mathematical history the discovery of the complex numbers lead to a mathematical breakthroughs. I only say "`fundamental theorem of algebra"' and "`complex analysis"' (which is a lot more elegant than "`standard calculus"' ;-) ). Since I think browser technology needs a radical breakthrough of disruptive changes Wolfgang Keller found it a fitting name. ;-)

Unluckily the project server of the university which he used at the beginning did not like the \^{} character in its name -- so he was requested to change the project name. Since he wanted to preserve the spirit of the name he chose the coefficient order of the polynomial $x^2+1$ which obviously is 1, 0, 1 and renamed it to 101\_browser.

\paragraph{What does the "`MTAx"' namespace mean?}

Since "`101"' is not a valid namespace in C++, "`MTAx"' was used, which is "`101"' encoded in Base64.

\part{User documentation}

\part{Reference documentation}

\part{Bits and pieces}

\chapter{Web standards}

\section{Sources for web standards}

\subsection{Crypto}

\subsubsection{SHA-1 and SHA-2}

\paragraph{Where can I get information concerning SHA-1 and SHA-2?}
\begin{itemize}
\item FIBS 180-3: \verb|http://csrc.nist.gov/publications/fips/fips180-3/fips180-3_final.pdf|
\item RFC 3174: \verb|http://www.apps.ietf.org/rfc/rfc3174.html|
\end{itemize}

\subsection{Graphics}

\subsubsection{GIF}

\paragraph{Where can I find the specification of GIF files?}
Look at \\
\verb|http://www.w3.org/Graphics/GIF/spec-gif87.txt| (GIF87A) and \\
\verb|http://www.w3.org/Graphics/GIF/spec-gif89a.txt| (GIF89A).

\subsubsection{JPEG}

\paragraph{Where can I get the ITU-T T.81 standard?} Under \verb|http://www.w3.org/Graphics/JPEG/| there is a link to \verb|http://www.w3.org/Graphics/JPEG/itu-t81.pdf|

\paragraph{Where can I get the ITU-T T.83 standard?} The authors of 101\_browser are not aware of a costless way that they are sure to be legal. So you will either have to pay or be on your own. :-(

Hint for adventurous people who don't fear legal consequences: search engines are a great invention. ;-)

\paragraph{Where can I obtain the test vectors for ITU-T T.83?} Under \\
\verb|http://www.itu.int/net/itu-t/sigdb/speimage/Tseries-s.htm|
there is a link to
\verb|http://www.itu.int/net/itu-t/sigdb/speimage/ImageForm-s.aspx?val=1010083|
where you can get the test vectors.

\subsubsection{PNG}

\paragraph{Where can I find the specification for PNG?} \verb|http://www.w3.org/TR/PNG/|

\paragraph{Where can I find a test suite for PNG?} \verb|http://www.schaik.com/pngsuite/| also mirrored on \verb|http://www.libpng.org/pub/png/pngsuite.html|

\subsection{Markup languages}

\subsubsection{WPF}

\paragraph{Where can I find specifications of WPF?} ~ \\
\verb|http://msdn.microsoft.com/en-us/library/ff629155(PROT.10).aspx|

\paragraph{Where can I find a test suite for WPF?} ~ \\
\verb|http://blogs.msdn.com/b/llobo/archive/2010/07/07/xaml-compliance-suite-v1.aspx|

\end{document}